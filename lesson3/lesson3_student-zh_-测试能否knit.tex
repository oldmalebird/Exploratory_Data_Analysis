\documentclass[]{article}
\usepackage{lmodern}
\usepackage{amssymb,amsmath}
\usepackage{ifxetex,ifluatex}
\usepackage{fixltx2e} % provides \textsubscript
\ifnum 0\ifxetex 1\fi\ifluatex 1\fi=0 % if pdftex
  \usepackage[T1]{fontenc}
  \usepackage[utf8]{inputenc}
\else % if luatex or xelatex
  \ifxetex
    \usepackage{mathspec}
  \else
    \usepackage{fontspec}
  \fi
  \defaultfontfeatures{Ligatures=TeX,Scale=MatchLowercase}
\fi
% use upquote if available, for straight quotes in verbatim environments
\IfFileExists{upquote.sty}{\usepackage{upquote}}{}
% use microtype if available
\IfFileExists{microtype.sty}{%
\usepackage{microtype}
\UseMicrotypeSet[protrusion]{basicmath} % disable protrusion for tt fonts
}{}
\usepackage[margin=1in]{geometry}
\usepackage{hyperref}
\hypersetup{unicode=true,
            pdfborder={0 0 0},
            breaklinks=true}
\urlstyle{same}  % don't use monospace font for urls
\usepackage{color}
\usepackage{fancyvrb}
\newcommand{\VerbBar}{|}
\newcommand{\VERB}{\Verb[commandchars=\\\{\}]}
\DefineVerbatimEnvironment{Highlighting}{Verbatim}{commandchars=\\\{\}}
% Add ',fontsize=\small' for more characters per line
\usepackage{framed}
\definecolor{shadecolor}{RGB}{248,248,248}
\newenvironment{Shaded}{\begin{snugshade}}{\end{snugshade}}
\newcommand{\AlertTok}[1]{\textcolor[rgb]{0.94,0.16,0.16}{#1}}
\newcommand{\AnnotationTok}[1]{\textcolor[rgb]{0.56,0.35,0.01}{\textbf{\textit{#1}}}}
\newcommand{\AttributeTok}[1]{\textcolor[rgb]{0.77,0.63,0.00}{#1}}
\newcommand{\BaseNTok}[1]{\textcolor[rgb]{0.00,0.00,0.81}{#1}}
\newcommand{\BuiltInTok}[1]{#1}
\newcommand{\CharTok}[1]{\textcolor[rgb]{0.31,0.60,0.02}{#1}}
\newcommand{\CommentTok}[1]{\textcolor[rgb]{0.56,0.35,0.01}{\textit{#1}}}
\newcommand{\CommentVarTok}[1]{\textcolor[rgb]{0.56,0.35,0.01}{\textbf{\textit{#1}}}}
\newcommand{\ConstantTok}[1]{\textcolor[rgb]{0.00,0.00,0.00}{#1}}
\newcommand{\ControlFlowTok}[1]{\textcolor[rgb]{0.13,0.29,0.53}{\textbf{#1}}}
\newcommand{\DataTypeTok}[1]{\textcolor[rgb]{0.13,0.29,0.53}{#1}}
\newcommand{\DecValTok}[1]{\textcolor[rgb]{0.00,0.00,0.81}{#1}}
\newcommand{\DocumentationTok}[1]{\textcolor[rgb]{0.56,0.35,0.01}{\textbf{\textit{#1}}}}
\newcommand{\ErrorTok}[1]{\textcolor[rgb]{0.64,0.00,0.00}{\textbf{#1}}}
\newcommand{\ExtensionTok}[1]{#1}
\newcommand{\FloatTok}[1]{\textcolor[rgb]{0.00,0.00,0.81}{#1}}
\newcommand{\FunctionTok}[1]{\textcolor[rgb]{0.00,0.00,0.00}{#1}}
\newcommand{\ImportTok}[1]{#1}
\newcommand{\InformationTok}[1]{\textcolor[rgb]{0.56,0.35,0.01}{\textbf{\textit{#1}}}}
\newcommand{\KeywordTok}[1]{\textcolor[rgb]{0.13,0.29,0.53}{\textbf{#1}}}
\newcommand{\NormalTok}[1]{#1}
\newcommand{\OperatorTok}[1]{\textcolor[rgb]{0.81,0.36,0.00}{\textbf{#1}}}
\newcommand{\OtherTok}[1]{\textcolor[rgb]{0.56,0.35,0.01}{#1}}
\newcommand{\PreprocessorTok}[1]{\textcolor[rgb]{0.56,0.35,0.01}{\textit{#1}}}
\newcommand{\RegionMarkerTok}[1]{#1}
\newcommand{\SpecialCharTok}[1]{\textcolor[rgb]{0.00,0.00,0.00}{#1}}
\newcommand{\SpecialStringTok}[1]{\textcolor[rgb]{0.31,0.60,0.02}{#1}}
\newcommand{\StringTok}[1]{\textcolor[rgb]{0.31,0.60,0.02}{#1}}
\newcommand{\VariableTok}[1]{\textcolor[rgb]{0.00,0.00,0.00}{#1}}
\newcommand{\VerbatimStringTok}[1]{\textcolor[rgb]{0.31,0.60,0.02}{#1}}
\newcommand{\WarningTok}[1]{\textcolor[rgb]{0.56,0.35,0.01}{\textbf{\textit{#1}}}}
\usepackage{graphicx,grffile}
\makeatletter
\def\maxwidth{\ifdim\Gin@nat@width>\linewidth\linewidth\else\Gin@nat@width\fi}
\def\maxheight{\ifdim\Gin@nat@height>\textheight\textheight\else\Gin@nat@height\fi}
\makeatother
% Scale images if necessary, so that they will not overflow the page
% margins by default, and it is still possible to overwrite the defaults
% using explicit options in \includegraphics[width, height, ...]{}
\setkeys{Gin}{width=\maxwidth,height=\maxheight,keepaspectratio}
\IfFileExists{parskip.sty}{%
\usepackage{parskip}
}{% else
\setlength{\parindent}{0pt}
\setlength{\parskip}{6pt plus 2pt minus 1pt}
}
\setlength{\emergencystretch}{3em}  % prevent overfull lines
\providecommand{\tightlist}{%
  \setlength{\itemsep}{0pt}\setlength{\parskip}{0pt}}
\setcounter{secnumdepth}{0}
% Redefines (sub)paragraphs to behave more like sections
\ifx\paragraph\undefined\else
\let\oldparagraph\paragraph
\renewcommand{\paragraph}[1]{\oldparagraph{#1}\mbox{}}
\fi
\ifx\subparagraph\undefined\else
\let\oldsubparagraph\subparagraph
\renewcommand{\subparagraph}[1]{\oldsubparagraph{#1}\mbox{}}
\fi

%%% Use protect on footnotes to avoid problems with footnotes in titles
\let\rmarkdownfootnote\footnote%
\def\footnote{\protect\rmarkdownfootnote}

%%% Change title format to be more compact
\usepackage{titling}

% Create subtitle command for use in maketitle
\newcommand{\subtitle}[1]{
  \posttitle{
    \begin{center}\large#1\end{center}
    }
}

\setlength{\droptitle}{-2em}

  \title{}
    \pretitle{\vspace{\droptitle}}
  \posttitle{}
    \author{}
    \preauthor{}\postauthor{}
    \date{}
    \predate{}\postdate{}
  

\begin{document}

\hypertarget{-3-}{%
\section{第 3 课}\label{-3-}}

\begin{center}\rule{0.5\linewidth}{\linethickness}\end{center}

\subsubsection{首先做什么?}

注释:

\begin{center}\rule{0.5\linewidth}{\linethickness}\end{center}

\hypertarget{facebook-}{%
\subsubsection{Facebook 匿名用户数据}\label{facebook-}}

注释:

\begin{Shaded}
\begin{Highlighting}[]
\NormalTok{pf =}\StringTok{ }\KeywordTok{read.delim}\NormalTok{(}\StringTok{'pseudo_facebook.tsv'}\NormalTok{)}
\CommentTok{#视频用的是这个:}
\CommentTok{#pf <- read.csv('pseudo_facebook.tsv', sep = '\textbackslash{}t')}
\KeywordTok{names}\NormalTok{(pf)}
\end{Highlighting}
\end{Shaded}

\begin{verbatim}
##  [1] "userid"                "age"                  
##  [3] "dob_day"               "dob_year"             
##  [5] "dob_month"             "gender"               
##  [7] "tenure"                "friend_count"         
##  [9] "friendships_initiated" "likes"                
## [11] "likes_received"        "mobile_likes"         
## [13] "mobile_likes_received" "www_likes"            
## [15] "www_likes_received"
\end{verbatim}

\begin{center}\rule{0.5\linewidth}{\linethickness}\end{center}

\subsubsection{用户生日的直方图}

注释:

\begin{Shaded}
\begin{Highlighting}[]
\KeywordTok{library}\NormalTok{(ggplot2)}
\KeywordTok{ggplot}\NormalTok{(}\KeywordTok{aes}\NormalTok{(}\DataTypeTok{x =}\NormalTok{ dob_day), }\DataTypeTok{data =}\NormalTok{ pf) }\OperatorTok{+}\StringTok{ }
\StringTok{  }\KeywordTok{geom_histogram}\NormalTok{(}\DataTypeTok{binwidth =} \DecValTok{1}\NormalTok{) }\OperatorTok{+}\StringTok{ }
\StringTok{  }\KeywordTok{scale_x_continuous}\NormalTok{(}\DataTypeTok{breaks =} \DecValTok{1}\OperatorTok{:}\DecValTok{31}\NormalTok{)}
\end{Highlighting}
\end{Shaded}

\includegraphics{lesson3_student-zh_-测试能否knit_files/figure-latex/unnamed-chunk-2-1.pdf}

\begin{Shaded}
\begin{Highlighting}[]
\KeywordTok{library}\NormalTok{(ggplot2)}
\KeywordTok{qplot}\NormalTok{(}\DataTypeTok{x =}\NormalTok{ dob_day, }\DataTypeTok{data =}\NormalTok{pf)}
\end{Highlighting}
\end{Shaded}

\begin{verbatim}
## `stat_bin()` using `bins = 30`. Pick better value with `binwidth`.
\end{verbatim}

\includegraphics{lesson3_student-zh_-测试能否knit_files/figure-latex/unnamed-chunk-3-1.pdf}

\begin{Shaded}
\begin{Highlighting}[]
\KeywordTok{qplot}\NormalTok{(}\DataTypeTok{x =}\NormalTok{ dob_day, }\DataTypeTok{data =}\NormalTok{pf) }\OperatorTok{+}
\StringTok{  }\KeywordTok{scale_x_continuous}\NormalTok{(}\DataTypeTok{breaks=}\DecValTok{1}\OperatorTok{:}\DecValTok{31}\NormalTok{)}
\end{Highlighting}
\end{Shaded}

\begin{verbatim}
## `stat_bin()` using `bins = 30`. Pick better value with `binwidth`.
\end{verbatim}

\includegraphics{lesson3_student-zh_-测试能否knit_files/figure-latex/unnamed-chunk-4-1.pdf}

\begin{center}\rule{0.5\linewidth}{\linethickness}\end{center}

\paragraph{在这个直方图中你注意到哪些事情?}

注册帐号时,很多网站会将你的生日默认为1号,甚至1月1号。对于
Facebook,他们的默认是1月1日。

\begin{center}\rule{0.5\linewidth}{\linethickness}\end{center}

\hypertarget{moira-}{%
\subsubsection{Moira 的调查}\label{moira-}}

注释: 人们感知的读者人数与实际读者人数相差巨大,人们严重低估读者人数。

\begin{center}\rule{0.5\linewidth}{\linethickness}\end{center}

\subsubsection{估算关注量}

注释:

\begin{center}\rule{0.5\linewidth}{\linethickness}\end{center}

\hypertarget{-facebook-}{%
\paragraph{思考一下,你在 Facebook
发布特定信息或分享图片的时间。什么时间呢?}\label{-facebook-}}

回复:

\paragraph{你认为多少朋友会看到你的发布?}

回复:

\hypertarget{-facebook-}{%
\paragraph{思考一下,你在 Facebook
每个月发布信息或进行评论的比例。你认为这个百分比是多少?}\label{-facebook-}}

回复:

\begin{center}\rule{0.5\linewidth}{\linethickness}\end{center}

\subsubsection{自我感知的关注量}

注释:

\begin{center}\rule{0.5\linewidth}{\linethickness}\end{center}

\subsubsection{拆分}

注释:

\begin{Shaded}
\begin{Highlighting}[]
\CommentTok{#只有一个变量时用facet_wrap,两个以上时用facet_grid}
\KeywordTok{qplot}\NormalTok{(}\DataTypeTok{x =}\NormalTok{ dob_day, }\DataTypeTok{data =}\NormalTok{ pf) }\OperatorTok{+}
\StringTok{  }\KeywordTok{scale_x_continuous}\NormalTok{(}\DataTypeTok{breaks=} \DecValTok{1}\OperatorTok{:}\DecValTok{31}\NormalTok{) }\OperatorTok{+}
\StringTok{  }\KeywordTok{facet_wrap}\NormalTok{(}\OperatorTok{~}\NormalTok{dob_month, }\DataTypeTok{ncol =} \DecValTok{3}\NormalTok{)}
\end{Highlighting}
\end{Shaded}

\begin{verbatim}
## `stat_bin()` using `bins = 30`. Pick better value with `binwidth`.
\end{verbatim}

\includegraphics{lesson3_student-zh_-测试能否knit_files/figure-latex/unnamed-chunk-5-1.pdf}

\begin{Shaded}
\begin{Highlighting}[]
\KeywordTok{ggplot}\NormalTok{(}\DataTypeTok{data =}\NormalTok{ pf, }\KeywordTok{aes}\NormalTok{(}\DataTypeTok{x =}\NormalTok{ dob_day)) }\OperatorTok{+}\StringTok{ }
\StringTok{  }\KeywordTok{geom_histogram}\NormalTok{(}\DataTypeTok{binwidth =} \DecValTok{1}\NormalTok{) }\OperatorTok{+}\StringTok{ }
\StringTok{  }\KeywordTok{scale_x_continuous}\NormalTok{(}\DataTypeTok{breaks =} \DecValTok{1}\OperatorTok{:}\DecValTok{31}\NormalTok{) }\OperatorTok{+}\StringTok{ }
\StringTok{  }\KeywordTok{facet_wrap}\NormalTok{(}\OperatorTok{~}\NormalTok{dob_month)}
\end{Highlighting}
\end{Shaded}

\includegraphics{lesson3_student-zh_-测试能否knit_files/figure-latex/unnamed-chunk-6-1.pdf}

\paragraph{我们再观察一下这个图。这里你发现了什么?}

1月1日的值很高

\begin{center}\rule{0.5\linewidth}{\linethickness}\end{center}

\subsubsection{保持怀疑态度:离群值和异常情况}

注释:

\begin{center}\rule{0.5\linewidth}{\linethickness}\end{center}

\hypertarget{moira-}{%
\subsubsection{Moira 的离群值}\label{moira-}}

注释:有人认为有1000万好友 \#\#\#\# 你认为哪些情况适用于 Moira
的离群值? bad data about an extreme case

\begin{center}\rule{0.5\linewidth}{\linethickness}\end{center}

\subsubsection{好友数}

注释:

\paragraph{你会输入哪个代码,创建朋友数量的直方图?}

\begin{Shaded}
\begin{Highlighting}[]
\KeywordTok{qplot}\NormalTok{(}\DataTypeTok{x =}\NormalTok{ friend_count, }\DataTypeTok{data =}\NormalTok{ pf)}
\end{Highlighting}
\end{Shaded}

\begin{verbatim}
## `stat_bin()` using `bins = 30`. Pick better value with `binwidth`.
\end{verbatim}

\includegraphics{lesson3_student-zh_-测试能否knit_files/figure-latex/unnamed-chunk-7-1.pdf}

\begin{Shaded}
\begin{Highlighting}[]
\CommentTok{#qplot(x = friend_count, data = pf, xlim = c(0,1000))}
\KeywordTok{qplot}\NormalTok{(}\DataTypeTok{x =}\NormalTok{ friend_count, }\DataTypeTok{data =}\NormalTok{ pf) }\OperatorTok{+}\StringTok{ }
\StringTok{  }\KeywordTok{scale_x_continuous}\NormalTok{(}\DataTypeTok{limits =} \KeywordTok{c}\NormalTok{(}\DecValTok{0}\NormalTok{,}\DecValTok{1000}\NormalTok{))}
\end{Highlighting}
\end{Shaded}

\begin{verbatim}
## `stat_bin()` using `bins = 30`. Pick better value with `binwidth`.
\end{verbatim}

\begin{verbatim}
## Warning: Removed 2951 rows containing non-finite values (stat_bin).
\end{verbatim}

\includegraphics{lesson3_student-zh_-测试能否knit_files/figure-latex/unnamed-chunk-8-1.pdf}

\hypertarget{-moira-}{%
\paragraph{这个图与 Moira 的第一个图有哪些相似的地方?}\label{-moira-}}

long tail data

\begin{center}\rule{0.5\linewidth}{\linethickness}\end{center}

\subsubsection{限制轴线}

注释:

\begin{Shaded}
\begin{Highlighting}[]
\KeywordTok{ggplot}\NormalTok{(}\KeywordTok{aes}\NormalTok{(}\DataTypeTok{x =}\NormalTok{ friend_count), }\DataTypeTok{data =}\NormalTok{ pf) }\OperatorTok{+}\StringTok{ }
\StringTok{  }\KeywordTok{geom_histogram}\NormalTok{() }\OperatorTok{+}\StringTok{ }
\KeywordTok{scale_x_continuous}\NormalTok{(}\DataTypeTok{limits =} \KeywordTok{c}\NormalTok{(}\DecValTok{0}\NormalTok{, }\DecValTok{1000}\NormalTok{))}
\end{Highlighting}
\end{Shaded}

\begin{verbatim}
## `stat_bin()` using `bins = 30`. Pick better value with `binwidth`.
\end{verbatim}

\begin{verbatim}
## Warning: Removed 2951 rows containing non-finite values (stat_bin).
\end{verbatim}

\includegraphics{lesson3_student-zh_-测试能否knit_files/figure-latex/unnamed-chunk-9-1.pdf}

\begin{Shaded}
\begin{Highlighting}[]
\KeywordTok{ggplot}\NormalTok{(}\KeywordTok{aes}\NormalTok{(}\DataTypeTok{x =}\NormalTok{ friend_count), }\DataTypeTok{data =}\NormalTok{ pf) }\OperatorTok{+}\StringTok{ }
\StringTok{  }\KeywordTok{geom_histogram}\NormalTok{() }\OperatorTok{+}\StringTok{ }
\KeywordTok{scale_x_continuous}\NormalTok{(}\DataTypeTok{limits =} \KeywordTok{c}\NormalTok{(}\DecValTok{0}\NormalTok{, }\DecValTok{1000}\NormalTok{))}
\end{Highlighting}
\end{Shaded}

\begin{verbatim}
## `stat_bin()` using `bins = 30`. Pick better value with `binwidth`.
\end{verbatim}

\begin{verbatim}
## Warning: Removed 2951 rows containing non-finite values (stat_bin).
\end{verbatim}

\includegraphics{lesson3_student-zh_-测试能否knit_files/figure-latex/unnamed-chunk-10-1.pdf}

\subsubsection{探究箱宽}

注释:

\begin{center}\rule{0.5\linewidth}{\linethickness}\end{center}

\subsubsection{调整箱宽}

注释:

\begin{Shaded}
\begin{Highlighting}[]
\KeywordTok{ggplot}\NormalTok{(}\KeywordTok{aes}\NormalTok{(}\DataTypeTok{x =}\NormalTok{ friend_count), }\DataTypeTok{data =}\NormalTok{ pf) }\OperatorTok{+}\StringTok{ }
\StringTok{  }\KeywordTok{geom_histogram}\NormalTok{(}\DataTypeTok{binwidth =} \DecValTok{25}\NormalTok{) }\OperatorTok{+}\StringTok{ }
\StringTok{  }\KeywordTok{scale_x_continuous}\NormalTok{(}\DataTypeTok{limits =} \KeywordTok{c}\NormalTok{(}\DecValTok{0}\NormalTok{, }\DecValTok{1000}\NormalTok{), }\DataTypeTok{breaks =} \KeywordTok{seq}\NormalTok{(}\DecValTok{0}\NormalTok{, }\DecValTok{1000}\NormalTok{, }\DecValTok{50}\NormalTok{))}
\end{Highlighting}
\end{Shaded}

\begin{verbatim}
## Warning: Removed 2951 rows containing non-finite values (stat_bin).
\end{verbatim}

\includegraphics{lesson3_student-zh_-测试能否knit_files/figure-latex/unnamed-chunk-11-1.pdf}

\begin{Shaded}
\begin{Highlighting}[]
\KeywordTok{qplot}\NormalTok{(}\DataTypeTok{x =}\NormalTok{ friend_count, }\DataTypeTok{data =}\NormalTok{ pf, }\DataTypeTok{binwidth =} \DecValTok{25}\NormalTok{) }\OperatorTok{+}\StringTok{ }
\StringTok{  }\KeywordTok{scale_x_continuous}\NormalTok{(}\DataTypeTok{limits =} \KeywordTok{c}\NormalTok{(}\DecValTok{0}\NormalTok{, }\DecValTok{1000}\NormalTok{), }\DataTypeTok{breaks =} \KeywordTok{seq}\NormalTok{(}\DecValTok{0}\NormalTok{, }\DecValTok{1000}\NormalTok{, }\DecValTok{50}\NormalTok{))}
\end{Highlighting}
\end{Shaded}

\begin{verbatim}
## Warning: Removed 2951 rows containing non-finite values (stat_bin).
\end{verbatim}

\includegraphics{lesson3_student-zh_-测试能否knit_files/figure-latex/unnamed-chunk-12-1.pdf}

\subsubsection{拆分好友数}

\begin{Shaded}
\begin{Highlighting}[]
\CommentTok{# 你会输入哪个代码,创建按照性别的直方图平面?}
\CommentTok{# 将它增加到下列代码中。}
\KeywordTok{qplot}\NormalTok{(}\DataTypeTok{x =}\NormalTok{ friend_count, }\DataTypeTok{data =}\NormalTok{ pf, }\DataTypeTok{binwidth =} \DecValTok{25}\NormalTok{) }\OperatorTok{+}
\StringTok{  }\KeywordTok{scale_x_continuous}\NormalTok{(}\DataTypeTok{limits =} \KeywordTok{c}\NormalTok{(}\DecValTok{0}\NormalTok{, }\DecValTok{1000}\NormalTok{),}
                     \DataTypeTok{breaks =} \KeywordTok{seq}\NormalTok{(}\DecValTok{0}\NormalTok{, }\DecValTok{1000}\NormalTok{, }\DecValTok{50}\NormalTok{)) }\OperatorTok{+}
\StringTok{  }\KeywordTok{facet_grid}\NormalTok{(. }\OperatorTok{~}\StringTok{ }\NormalTok{gender) }
\end{Highlighting}
\end{Shaded}

\begin{verbatim}
## Warning: Removed 2951 rows containing non-finite values (stat_bin).
\end{verbatim}

\includegraphics{lesson3_student-zh_-测试能否knit_files/figure-latex/unnamed-chunk-13-1.pdf}

\begin{Shaded}
\begin{Highlighting}[]
\KeywordTok{ggplot}\NormalTok{(}\KeywordTok{aes}\NormalTok{(}\DataTypeTok{x =}\NormalTok{ friend_count), }\DataTypeTok{data =}\NormalTok{ pf) }\OperatorTok{+}\StringTok{ }
\StringTok{  }\KeywordTok{geom_histogram}\NormalTok{() }\OperatorTok{+}\StringTok{ }
\StringTok{  }\KeywordTok{scale_x_continuous}\NormalTok{(}\DataTypeTok{limits =} \KeywordTok{c}\NormalTok{(}\DecValTok{0}\NormalTok{, }\DecValTok{1000}\NormalTok{), }\DataTypeTok{breaks =} \KeywordTok{seq}\NormalTok{(}\DecValTok{0}\NormalTok{, }\DecValTok{1000}\NormalTok{, }\DecValTok{50}\NormalTok{)) }\OperatorTok{+}\StringTok{ }
\StringTok{  }\KeywordTok{facet_wrap}\NormalTok{(}\OperatorTok{~}\NormalTok{gender)}
\end{Highlighting}
\end{Shaded}

\begin{verbatim}
## `stat_bin()` using `bins = 30`. Pick better value with `binwidth`.
\end{verbatim}

\begin{verbatim}
## Warning: Removed 2951 rows containing non-finite values (stat_bin).
\end{verbatim}

\includegraphics{lesson3_student-zh_-测试能否knit_files/figure-latex/unnamed-chunk-14-1.pdf}

\begin{center}\rule{0.5\linewidth}{\linethickness}\end{center}

\subsubsection{省略不适用的数值}

注释:如果用了data = na.omit(pf),则其他列的空值也被去掉了

\begin{Shaded}
\begin{Highlighting}[]
\KeywordTok{qplot}\NormalTok{(}\DataTypeTok{x =}\NormalTok{ friend_count, }\DataTypeTok{data =} \KeywordTok{subset}\NormalTok{(pf,}\OperatorTok{!}\KeywordTok{is.na}\NormalTok{(gender)), }\DataTypeTok{binwidth =} \DecValTok{10}\NormalTok{) }\OperatorTok{+}
\StringTok{  }\KeywordTok{scale_x_continuous}\NormalTok{(}\DataTypeTok{limits =} \KeywordTok{c}\NormalTok{(}\DecValTok{0}\NormalTok{, }\DecValTok{1000}\NormalTok{), }\DataTypeTok{breaks =} \KeywordTok{seq}\NormalTok{(}\DecValTok{0}\NormalTok{, }\DecValTok{1000}\NormalTok{, }\DecValTok{50}\NormalTok{)) }\OperatorTok{+}
\StringTok{  }\KeywordTok{facet_wrap}\NormalTok{(. }\OperatorTok{~}\StringTok{ }\NormalTok{gender) }
\end{Highlighting}
\end{Shaded}

\begin{verbatim}
## Warning: Removed 2949 rows containing non-finite values (stat_bin).
\end{verbatim}

\includegraphics{lesson3_student-zh_-测试能否knit_files/figure-latex/unnamed-chunk-15-1.pdf}

\begin{Shaded}
\begin{Highlighting}[]
\KeywordTok{ggplot}\NormalTok{(}\KeywordTok{aes}\NormalTok{(}\DataTypeTok{x =}\NormalTok{ friend_count), }\DataTypeTok{data =} \KeywordTok{subset}\NormalTok{(pf, }\OperatorTok{!}\KeywordTok{is.na}\NormalTok{(gender))) }\OperatorTok{+}\StringTok{ }
\StringTok{  }\KeywordTok{geom_histogram}\NormalTok{() }\OperatorTok{+}\StringTok{ }
\StringTok{  }\KeywordTok{scale_x_continuous}\NormalTok{(}\DataTypeTok{limits =} \KeywordTok{c}\NormalTok{(}\DecValTok{0}\NormalTok{, }\DecValTok{1000}\NormalTok{), }\DataTypeTok{breaks =} \KeywordTok{seq}\NormalTok{(}\DecValTok{0}\NormalTok{, }\DecValTok{1000}\NormalTok{, }\DecValTok{50}\NormalTok{)) }\OperatorTok{+}\StringTok{ }
\StringTok{  }\KeywordTok{facet_wrap}\NormalTok{(}\OperatorTok{~}\NormalTok{gender)}
\end{Highlighting}
\end{Shaded}

\begin{verbatim}
## `stat_bin()` using `bins = 30`. Pick better value with `binwidth`.
\end{verbatim}

\begin{verbatim}
## Warning: Removed 2949 rows containing non-finite values (stat_bin).
\end{verbatim}

\includegraphics{lesson3_student-zh_-测试能否knit_files/figure-latex/unnamed-chunk-16-1.pdf}

\begin{center}\rule{0.5\linewidth}{\linethickness}\end{center}

\hypertarget{-}{%
\subsubsection{`根据' 性别的统计量}\label{-}}

注释:

\begin{Shaded}
\begin{Highlighting}[]
\KeywordTok{table}\NormalTok{(pf}\OperatorTok{$}\NormalTok{gender)}
\end{Highlighting}
\end{Shaded}

\begin{verbatim}
## 
## female   male 
##  40254  58574
\end{verbatim}

\begin{Shaded}
\begin{Highlighting}[]
\KeywordTok{table}\NormalTok{(pf}\OperatorTok{$}\NormalTok{gender)}
\end{Highlighting}
\end{Shaded}

\begin{verbatim}
## 
## female   male 
##  40254  58574
\end{verbatim}

\begin{Shaded}
\begin{Highlighting}[]
\KeywordTok{by}\NormalTok{(pf}\OperatorTok{$}\NormalTok{friend_count,pf}\OperatorTok{$}\NormalTok{gender, summary)}
\end{Highlighting}
\end{Shaded}

\begin{verbatim}
## pf$gender: female
##    Min. 1st Qu.  Median    Mean 3rd Qu.    Max. 
##       0      37      96     242     244    4923 
## -------------------------------------------------------- 
## pf$gender: male
##    Min. 1st Qu.  Median    Mean 3rd Qu.    Max. 
##       0      27      74     165     182    4917
\end{verbatim}

\paragraph{哪些人朋友更多,男性还是女性?}

回复:女性

\paragraph{女性和男性的朋友数量中位数有哪些不同?}

回复:96,74

\paragraph{为什么中位数是比平均数更好的测量方法?}

回复:受极值影响小

\begin{center}\rule{0.5\linewidth}{\linethickness}\end{center}

\subsubsection{使用时长}

注释:

\begin{Shaded}
\begin{Highlighting}[]
\KeywordTok{qplot}\NormalTok{(}\DataTypeTok{x =}\NormalTok{ tenure, }\DataTypeTok{data =}\NormalTok{ pf, }\DataTypeTok{binwidth =} \DecValTok{30}\NormalTok{, }\DataTypeTok{color =} \KeywordTok{I}\NormalTok{(}\StringTok{'black'}\NormalTok{), }\DataTypeTok{fill =} \KeywordTok{I}\NormalTok{(}\StringTok{'#099DD9'}\NormalTok{))}
\end{Highlighting}
\end{Shaded}

\begin{verbatim}
## Warning: Removed 2 rows containing non-finite values (stat_bin).
\end{verbatim}

\includegraphics{lesson3_student-zh_-测试能否knit_files/figure-latex/unnamed-chunk-19-1.pdf}

\begin{Shaded}
\begin{Highlighting}[]
\KeywordTok{ggplot}\NormalTok{(}\KeywordTok{aes}\NormalTok{(}\DataTypeTok{x =}\NormalTok{ tenure), }\DataTypeTok{data =}\NormalTok{ pf) }\OperatorTok{+}\StringTok{ }
\StringTok{   }\KeywordTok{geom_histogram}\NormalTok{(}\DataTypeTok{binwidth =} \DecValTok{30}\NormalTok{, }\DataTypeTok{color =} \StringTok{'black'}\NormalTok{, }\DataTypeTok{fill =} \StringTok{'#099DD9'}\NormalTok{)}
\end{Highlighting}
\end{Shaded}

\begin{verbatim}
## Warning: Removed 2 rows containing non-finite values (stat_bin).
\end{verbatim}

\includegraphics{lesson3_student-zh_-测试能否knit_files/figure-latex/unnamed-chunk-20-1.pdf}
***

\paragraph{你如何按照年份创建任期的直方图?}

\begin{Shaded}
\begin{Highlighting}[]
\KeywordTok{ggplot}\NormalTok{(}\KeywordTok{aes}\NormalTok{(}\DataTypeTok{x =}\NormalTok{ tenure}\OperatorTok{/}\DecValTok{365}\NormalTok{), }\DataTypeTok{data =}\NormalTok{ pf) }\OperatorTok{+}\StringTok{ }
\StringTok{   }\KeywordTok{geom_histogram}\NormalTok{(}\DataTypeTok{binwidth =} \FloatTok{.25}\NormalTok{, }\DataTypeTok{color =} \StringTok{'black'}\NormalTok{, }\DataTypeTok{fill =} \StringTok{'#F79420'}\NormalTok{)}
\end{Highlighting}
\end{Shaded}

\begin{verbatim}
## Warning: Removed 2 rows containing non-finite values (stat_bin).
\end{verbatim}

\includegraphics{lesson3_student-zh_-测试能否knit_files/figure-latex/unnamed-chunk-21-1.pdf}

\begin{Shaded}
\begin{Highlighting}[]
\KeywordTok{qplot}\NormalTok{(}\DataTypeTok{x =}\NormalTok{ tenure}\OperatorTok{/}\DecValTok{365}\NormalTok{, }\DataTypeTok{data =}\NormalTok{ pf, }\DataTypeTok{binwidth =} \FloatTok{.25}\NormalTok{, }\DataTypeTok{color =} \KeywordTok{I}\NormalTok{(}\StringTok{'black'}\NormalTok{), }\DataTypeTok{fill =} \KeywordTok{I}\NormalTok{(}\StringTok{'#F79420'}\NormalTok{)) }\OperatorTok{+}
\StringTok{  }\KeywordTok{scale_x_continuous}\NormalTok{(}\DataTypeTok{breaks =} \KeywordTok{seq}\NormalTok{(}\DecValTok{1}\NormalTok{,}\DecValTok{7}\NormalTok{,}\DecValTok{1}\NormalTok{), }\DataTypeTok{limits =} \KeywordTok{c}\NormalTok{(}\DecValTok{0}\NormalTok{,}\DecValTok{7}\NormalTok{))}
\end{Highlighting}
\end{Shaded}

\begin{verbatim}
## Warning: Removed 26 rows containing non-finite values (stat_bin).
\end{verbatim}

\includegraphics{lesson3_student-zh_-测试能否knit_files/figure-latex/unnamed-chunk-22-1.pdf}

\begin{center}\rule{0.5\linewidth}{\linethickness}\end{center}

\subsubsection{标签图}

注释:

\begin{Shaded}
\begin{Highlighting}[]
\KeywordTok{qplot}\NormalTok{(}\DataTypeTok{x =}\NormalTok{ tenure}\OperatorTok{/}\DecValTok{365}\NormalTok{, }\DataTypeTok{data =}\NormalTok{ pf, }
      \DataTypeTok{xlab =} \StringTok{'Number of years using Facebook'}\NormalTok{,}
      \DataTypeTok{ylab =} \StringTok{'Number of users in sample'}\NormalTok{,}
      \DataTypeTok{color =} \KeywordTok{I}\NormalTok{(}\StringTok{'black'}\NormalTok{), }\DataTypeTok{fill =} \KeywordTok{I}\NormalTok{(}\StringTok{'#F79420'}\NormalTok{)) }\OperatorTok{+}
\StringTok{  }\KeywordTok{scale_x_continuous}\NormalTok{(}\DataTypeTok{breaks =} \KeywordTok{seq}\NormalTok{(}\DecValTok{1}\NormalTok{,}\DecValTok{7}\NormalTok{,}\DecValTok{1}\NormalTok{), }\DataTypeTok{limits =} \KeywordTok{c}\NormalTok{(}\DecValTok{0}\NormalTok{,}\DecValTok{7}\NormalTok{))}
\end{Highlighting}
\end{Shaded}

\begin{verbatim}
## `stat_bin()` using `bins = 30`. Pick better value with `binwidth`.
\end{verbatim}

\begin{verbatim}
## Warning: Removed 26 rows containing non-finite values (stat_bin).
\end{verbatim}

\includegraphics{lesson3_student-zh_-测试能否knit_files/figure-latex/unnamed-chunk-23-1.pdf}

\begin{center}\rule{0.5\linewidth}{\linethickness}\end{center}

\subsubsection{用户年龄}

注释:

\begin{Shaded}
\begin{Highlighting}[]
\KeywordTok{qplot}\NormalTok{(}\DataTypeTok{x =}\NormalTok{ age, }\DataTypeTok{data =}\NormalTok{ pf, }\DataTypeTok{binwidth =}\DecValTok{1}\NormalTok{,}
      \DataTypeTok{xlab =} \StringTok{'Age'}\NormalTok{,}
      \DataTypeTok{ylab =} \StringTok{'Number of users in sample'}\NormalTok{,}
      \DataTypeTok{color =} \KeywordTok{I}\NormalTok{(}\StringTok{'black'}\NormalTok{), }\DataTypeTok{fill =} \KeywordTok{I}\NormalTok{(}\StringTok{'#5760AB'}\NormalTok{)) }\OperatorTok{+}
\StringTok{  }\KeywordTok{scale_x_continuous}\NormalTok{(}\DataTypeTok{breaks =} \KeywordTok{seq}\NormalTok{(}\DecValTok{0}\NormalTok{,}\DecValTok{113}\NormalTok{,}\DecValTok{5}\NormalTok{))}
\end{Highlighting}
\end{Shaded}

\includegraphics{lesson3_student-zh_-测试能否knit_files/figure-latex/unnamed-chunk-24-1.pdf}

\begin{Shaded}
\begin{Highlighting}[]
\KeywordTok{ggplot}\NormalTok{(}\KeywordTok{aes}\NormalTok{(}\DataTypeTok{x =}\NormalTok{ age), }\DataTypeTok{data =}\NormalTok{ pf) }\OperatorTok{+}\StringTok{ }
\StringTok{  }\KeywordTok{geom_histogram}\NormalTok{(}\DataTypeTok{binwidth =} \DecValTok{1}\NormalTok{, }\DataTypeTok{fill =} \StringTok{'#5760AB'}\NormalTok{) }\OperatorTok{+}\StringTok{ }
\StringTok{  }\KeywordTok{scale_x_continuous}\NormalTok{(}\DataTypeTok{breaks =} \KeywordTok{seq}\NormalTok{(}\DecValTok{0}\NormalTok{, }\DecValTok{113}\NormalTok{, }\DecValTok{5}\NormalTok{))}
\end{Highlighting}
\end{Shaded}

\includegraphics{lesson3_student-zh_-测试能否knit_files/figure-latex/unnamed-chunk-25-1.pdf}

\paragraph{你发现了什么?}

回复:100+

\begin{center}\rule{0.5\linewidth}{\linethickness}\end{center}

\subsubsection{表情包的传播}

注释:

\begin{center}\rule{0.5\linewidth}{\linethickness}\end{center}

\hypertarget{lada-}{%
\subsubsection{Lada 钱包表情包}\label{lada-}}

注释:

\begin{center}\rule{0.5\linewidth}{\linethickness}\end{center}

\subsubsection{改变数据}

注释:

\begin{Shaded}
\begin{Highlighting}[]
\KeywordTok{summary}\NormalTok{(pf}\OperatorTok{$}\NormalTok{friend_count)}
\end{Highlighting}
\end{Shaded}

\begin{verbatim}
##    Min. 1st Qu.  Median    Mean 3rd Qu.    Max. 
##     0.0    31.0    82.0   196.4   206.0  4923.0
\end{verbatim}

\begin{Shaded}
\begin{Highlighting}[]
\KeywordTok{summary}\NormalTok{(}\KeywordTok{log10}\NormalTok{(pf}\OperatorTok{$}\NormalTok{friend_count))}
\end{Highlighting}
\end{Shaded}

\begin{verbatim}
##    Min. 1st Qu.  Median    Mean 3rd Qu.    Max. 
##    -Inf   1.491   1.914    -Inf   2.314   3.692
\end{verbatim}

\begin{Shaded}
\begin{Highlighting}[]
\KeywordTok{summary}\NormalTok{(}\KeywordTok{log10}\NormalTok{(pf}\OperatorTok{$}\NormalTok{friend_count }\OperatorTok{+}\DecValTok{1}\NormalTok{))}
\end{Highlighting}
\end{Shaded}

\begin{verbatim}
##    Min. 1st Qu.  Median    Mean 3rd Qu.    Max. 
##   0.000   1.505   1.919   1.868   2.316   3.692
\end{verbatim}

\begin{Shaded}
\begin{Highlighting}[]
\KeywordTok{summary}\NormalTok{(}\KeywordTok{sqrt}\NormalTok{(pf}\OperatorTok{$}\NormalTok{friend_count))}
\end{Highlighting}
\end{Shaded}

\begin{verbatim}
##    Min. 1st Qu.  Median    Mean 3rd Qu.    Max. 
##   0.000   5.568   9.055  11.088  14.353  70.164
\end{verbatim}

\begin{Shaded}
\begin{Highlighting}[]
\KeywordTok{library}\NormalTok{(gridExtra)}
\NormalTok{p1 <-}\StringTok{ }\KeywordTok{qplot}\NormalTok{(}\DataTypeTok{x =}\NormalTok{ friend_count, }\DataTypeTok{data =}\NormalTok{ pf)}
\NormalTok{p2 <-}\StringTok{ }\KeywordTok{qplot}\NormalTok{(}\DataTypeTok{x =} \KeywordTok{log10}\NormalTok{(friend_count }\OperatorTok{+}\StringTok{ }\DecValTok{1}\NormalTok{), }\DataTypeTok{data =}\NormalTok{ pf)}
\NormalTok{p3 <-}\StringTok{ }\KeywordTok{qplot}\NormalTok{(}\DataTypeTok{x =} \KeywordTok{sqrt}\NormalTok{(friend_count), }\DataTypeTok{data =}\NormalTok{ pf) }
\KeywordTok{grid.arrange}\NormalTok{(p1, p2, p3, }\DataTypeTok{ncol=}\DecValTok{1}\NormalTok{)}
\end{Highlighting}
\end{Shaded}

\begin{verbatim}
## `stat_bin()` using `bins = 30`. Pick better value with `binwidth`.
## `stat_bin()` using `bins = 30`. Pick better value with `binwidth`.
## `stat_bin()` using `bins = 30`. Pick better value with `binwidth`.
\end{verbatim}

\includegraphics{lesson3_student-zh_-测试能否knit_files/figure-latex/unnamed-chunk-27-1.pdf}

\begin{Shaded}
\begin{Highlighting}[]
\NormalTok{p1 <-}\StringTok{ }\KeywordTok{ggplot}\NormalTok{(}\KeywordTok{aes}\NormalTok{(}\DataTypeTok{x =}\NormalTok{ friend_count), }\DataTypeTok{data =}\NormalTok{ pf) }\OperatorTok{+}\StringTok{ }\KeywordTok{geom_histogram}\NormalTok{()}
\NormalTok{p2 <-}\StringTok{ }\NormalTok{p1 }\OperatorTok{+}\StringTok{ }\KeywordTok{scale_x_log10}\NormalTok{()}
\NormalTok{p3 <-}\StringTok{ }\NormalTok{p1 }\OperatorTok{+}\StringTok{ }\KeywordTok{scale_x_sqrt}\NormalTok{()}
\KeywordTok{grid.arrange}\NormalTok{(p1, p2, p3, }\DataTypeTok{ncol=}\DecValTok{1}\NormalTok{)}
\end{Highlighting}
\end{Shaded}

\begin{verbatim}
## `stat_bin()` using `bins = 30`. Pick better value with `binwidth`.
\end{verbatim}

\begin{verbatim}
## Warning: Transformation introduced infinite values in continuous x-axis
\end{verbatim}

\begin{verbatim}
## `stat_bin()` using `bins = 30`. Pick better value with `binwidth`.
\end{verbatim}

\begin{verbatim}
## Warning: Removed 1962 rows containing non-finite values (stat_bin).
\end{verbatim}

\begin{verbatim}
## `stat_bin()` using `bins = 30`. Pick better value with `binwidth`.
\end{verbatim}

\includegraphics{lesson3_student-zh_-测试能否knit_files/figure-latex/unnamed-chunk-28-1.pdf}
***

\subsubsection{调整尺寸图层}

注释:

\begin{Shaded}
\begin{Highlighting}[]
\NormalTok{logScale <-}\StringTok{ }\KeywordTok{qplot}\NormalTok{(}\DataTypeTok{x =} \KeywordTok{log10}\NormalTok{(friend_count), }\DataTypeTok{data =}\NormalTok{ pf)}
\NormalTok{countScale <-}\StringTok{ }\KeywordTok{ggplot}\NormalTok{(}\KeywordTok{aes}\NormalTok{(}\DataTypeTok{x =}\NormalTok{ friend_count), }\DataTypeTok{data =}\NormalTok{ pf) }\OperatorTok{+}\StringTok{ }
\StringTok{  }\KeywordTok{geom_histogram}\NormalTok{() }\OperatorTok{+}\StringTok{ }
\StringTok{  }\KeywordTok{scale_x_log10}\NormalTok{()}
\KeywordTok{grid.arrange}\NormalTok{(logScale, countScale, }\DataTypeTok{ncol =} \DecValTok{2}\NormalTok{)}
\end{Highlighting}
\end{Shaded}

\begin{verbatim}
## `stat_bin()` using `bins = 30`. Pick better value with `binwidth`.
\end{verbatim}

\begin{verbatim}
## Warning: Removed 1962 rows containing non-finite values (stat_bin).
\end{verbatim}

\begin{verbatim}
## Warning: Transformation introduced infinite values in continuous x-axis
\end{verbatim}

\begin{verbatim}
## `stat_bin()` using `bins = 30`. Pick better value with `binwidth`.
\end{verbatim}

\begin{verbatim}
## Warning: Removed 1962 rows containing non-finite values (stat_bin).
\end{verbatim}

\includegraphics{lesson3_student-zh_-测试能否knit_files/figure-latex/unnamed-chunk-29-1.pdf}

\begin{center}\rule{0.5\linewidth}{\linethickness}\end{center}

\subsubsection{频数多边图}

\begin{Shaded}
\begin{Highlighting}[]
\KeywordTok{qplot}\NormalTok{(}\DataTypeTok{x =}\NormalTok{ friend_count, }\DataTypeTok{data =} \KeywordTok{subset}\NormalTok{(pf,}\OperatorTok{!}\KeywordTok{is.na}\NormalTok{(gender)), }\DataTypeTok{binwidth =} \DecValTok{10}\NormalTok{,}
      \DataTypeTok{geom =} \StringTok{'freqpoly'}\NormalTok{, }\DataTypeTok{color =}\NormalTok{ gender) }\OperatorTok{+}
\StringTok{  }\KeywordTok{scale_x_continuous}\NormalTok{(}\DataTypeTok{limits =} \KeywordTok{c}\NormalTok{(}\DecValTok{0}\NormalTok{, }\DecValTok{1000}\NormalTok{), }\DataTypeTok{breaks =} \KeywordTok{seq}\NormalTok{(}\DecValTok{0}\NormalTok{, }\DecValTok{1000}\NormalTok{, }\DecValTok{50}\NormalTok{))}
\end{Highlighting}
\end{Shaded}

\begin{verbatim}
## Warning: Removed 2949 rows containing non-finite values (stat_bin).
\end{verbatim}

\begin{verbatim}
## Warning: Removed 4 rows containing missing values (geom_path).
\end{verbatim}

\includegraphics{lesson3_student-zh_-测试能否knit_files/figure-latex/unnamed-chunk-30-1.pdf}

\begin{Shaded}
\begin{Highlighting}[]
\KeywordTok{qplot}\NormalTok{(}\DataTypeTok{x =}\NormalTok{ friend_count, }\DataTypeTok{y =}\NormalTok{ ..count..}\OperatorTok{/}\KeywordTok{sum}\NormalTok{(..count..),}
      \DataTypeTok{data =} \KeywordTok{subset}\NormalTok{(pf,}\OperatorTok{!}\KeywordTok{is.na}\NormalTok{(gender)), }
      \DataTypeTok{xlab =} \StringTok{'Friend Count'}\NormalTok{,}
      \DataTypeTok{ylab =} \StringTok{'Proportion of Users with that friend count'}\NormalTok{,}
      \DataTypeTok{binwidth =} \DecValTok{10}\NormalTok{,}
      \DataTypeTok{geom =} \StringTok{'freqpoly'}\NormalTok{, }\DataTypeTok{color =}\NormalTok{ gender) }\OperatorTok{+}
\StringTok{  }\KeywordTok{scale_x_continuous}\NormalTok{(}\DataTypeTok{limits =} \KeywordTok{c}\NormalTok{(}\DecValTok{0}\NormalTok{, }\DecValTok{1000}\NormalTok{), }\DataTypeTok{breaks =} \KeywordTok{seq}\NormalTok{(}\DecValTok{0}\NormalTok{, }\DecValTok{1000}\NormalTok{, }\DecValTok{50}\NormalTok{))}
\end{Highlighting}
\end{Shaded}

\begin{verbatim}
## Warning: Removed 2949 rows containing non-finite values (stat_bin).
\end{verbatim}

\begin{verbatim}
## Warning: Removed 4 rows containing missing values (geom_path).
\end{verbatim}

\includegraphics{lesson3_student-zh_-测试能否knit_files/figure-latex/unnamed-chunk-31-1.pdf}

\begin{Shaded}
\begin{Highlighting}[]
\KeywordTok{ggplot}\NormalTok{(}\KeywordTok{aes}\NormalTok{(}\DataTypeTok{x =}\NormalTok{ friend_count, }\DataTypeTok{y =}\NormalTok{ ..count..}\OperatorTok{/}\KeywordTok{sum}\NormalTok{(..count..)), }\DataTypeTok{data =} \KeywordTok{subset}\NormalTok{(pf, }\OperatorTok{!}\KeywordTok{is.na}\NormalTok{(gender))) }\OperatorTok{+}
\StringTok{  }\KeywordTok{geom_freqpoly}\NormalTok{(}\KeywordTok{aes}\NormalTok{(}\DataTypeTok{color =}\NormalTok{ gender), }\DataTypeTok{binwidth=}\DecValTok{10}\NormalTok{) }\OperatorTok{+}\StringTok{ }
\StringTok{  }\KeywordTok{scale_x_continuous}\NormalTok{(}\DataTypeTok{limits =} \KeywordTok{c}\NormalTok{(}\DecValTok{0}\NormalTok{, }\DecValTok{1000}\NormalTok{), }\DataTypeTok{breaks =} \KeywordTok{seq}\NormalTok{(}\DecValTok{0}\NormalTok{, }\DecValTok{1000}\NormalTok{, }\DecValTok{50}\NormalTok{)) }\OperatorTok{+}\StringTok{ }
\StringTok{  }\KeywordTok{xlab}\NormalTok{(}\StringTok{'好友数量'}\NormalTok{) }\OperatorTok{+}\StringTok{ }
\StringTok{  }\KeywordTok{ylab}\NormalTok{(}\StringTok{'Percentage of users with that friend count'}\NormalTok{)}
\end{Highlighting}
\end{Shaded}

\begin{verbatim}
## Warning: Removed 2949 rows containing non-finite values (stat_bin).
\end{verbatim}

\begin{verbatim}
## Warning: Removed 4 rows containing missing values (geom_path).
\end{verbatim}

\begin{verbatim}
## Warning in grid.Call(C_textBounds, as.graphicsAnnot(x$label), x$x, x$y, :
## 'mbcsToSbcs'里转换'濂藉弸鏁伴噺'出错:<e5>代替了dot
\end{verbatim}

\begin{verbatim}
## Warning in grid.Call(C_textBounds, as.graphicsAnnot(x$label), x$x, x$y, :
## 'mbcsToSbcs'里转换'濂藉弸鏁伴噺'出错:<a5>代替了dot
\end{verbatim}

\begin{verbatim}
## Warning in grid.Call(C_textBounds, as.graphicsAnnot(x$label), x$x, x$y, :
## 'mbcsToSbcs'里转换'濂藉弸鏁伴噺'出错:<bd>代替了dot
\end{verbatim}

\begin{verbatim}
## Warning in grid.Call(C_textBounds, as.graphicsAnnot(x$label), x$x, x$y, :
## 'mbcsToSbcs'里转换'濂藉弸鏁伴噺'出错:<e5>代替了dot
\end{verbatim}

\begin{verbatim}
## Warning in grid.Call(C_textBounds, as.graphicsAnnot(x$label), x$x, x$y, :
## 'mbcsToSbcs'里转换'濂藉弸鏁伴噺'出错:<8f>代替了dot
\end{verbatim}

\begin{verbatim}
## Warning in grid.Call(C_textBounds, as.graphicsAnnot(x$label), x$x, x$y, :
## 'mbcsToSbcs'里转换'濂藉弸鏁伴噺'出错:<8b>代替了dot
\end{verbatim}

\begin{verbatim}
## Warning in grid.Call(C_textBounds, as.graphicsAnnot(x$label), x$x, x$y, :
## 'mbcsToSbcs'里转换'濂藉弸鏁伴噺'出错:<e6>代替了dot
\end{verbatim}

\begin{verbatim}
## Warning in grid.Call(C_textBounds, as.graphicsAnnot(x$label), x$x, x$y, :
## 'mbcsToSbcs'里转换'濂藉弸鏁伴噺'出错:<95>代替了dot
\end{verbatim}

\begin{verbatim}
## Warning in grid.Call(C_textBounds, as.graphicsAnnot(x$label), x$x, x$y, :
## 'mbcsToSbcs'里转换'濂藉弸鏁伴噺'出错:<b0>代替了dot
\end{verbatim}

\begin{verbatim}
## Warning in grid.Call(C_textBounds, as.graphicsAnnot(x$label), x$x, x$y, :
## 'mbcsToSbcs'里转换'濂藉弸鏁伴噺'出错:<e9>代替了dot
\end{verbatim}

\begin{verbatim}
## Warning in grid.Call(C_textBounds, as.graphicsAnnot(x$label), x$x, x$y, :
## 'mbcsToSbcs'里转换'濂藉弸鏁伴噺'出错:<87>代替了dot
\end{verbatim}

\begin{verbatim}
## Warning in grid.Call(C_textBounds, as.graphicsAnnot(x$label), x$x, x$y, :
## 'mbcsToSbcs'里转换'濂藉弸鏁伴噺'出错:<8f>代替了dot
\end{verbatim}

\begin{verbatim}
## Warning in grid.Call(C_textBounds, as.graphicsAnnot(x$label), x$x, x$y, :
## 'mbcsToSbcs'里转换'濂藉弸鏁伴噺'出错:<e5>代替了dot
\end{verbatim}

\begin{verbatim}
## Warning in grid.Call(C_textBounds, as.graphicsAnnot(x$label), x$x, x$y, :
## 'mbcsToSbcs'里转换'濂藉弸鏁伴噺'出错:<a5>代替了dot
\end{verbatim}

\begin{verbatim}
## Warning in grid.Call(C_textBounds, as.graphicsAnnot(x$label), x$x, x$y, :
## 'mbcsToSbcs'里转换'濂藉弸鏁伴噺'出错:<bd>代替了dot
\end{verbatim}

\begin{verbatim}
## Warning in grid.Call(C_textBounds, as.graphicsAnnot(x$label), x$x, x$y, :
## 'mbcsToSbcs'里转换'濂藉弸鏁伴噺'出错:<e5>代替了dot
\end{verbatim}

\begin{verbatim}
## Warning in grid.Call(C_textBounds, as.graphicsAnnot(x$label), x$x, x$y, :
## 'mbcsToSbcs'里转换'濂藉弸鏁伴噺'出错:<8f>代替了dot
\end{verbatim}

\begin{verbatim}
## Warning in grid.Call(C_textBounds, as.graphicsAnnot(x$label), x$x, x$y, :
## 'mbcsToSbcs'里转换'濂藉弸鏁伴噺'出错:<8b>代替了dot
\end{verbatim}

\begin{verbatim}
## Warning in grid.Call(C_textBounds, as.graphicsAnnot(x$label), x$x, x$y, :
## 'mbcsToSbcs'里转换'濂藉弸鏁伴噺'出错:<e6>代替了dot
\end{verbatim}

\begin{verbatim}
## Warning in grid.Call(C_textBounds, as.graphicsAnnot(x$label), x$x, x$y, :
## 'mbcsToSbcs'里转换'濂藉弸鏁伴噺'出错:<95>代替了dot
\end{verbatim}

\begin{verbatim}
## Warning in grid.Call(C_textBounds, as.graphicsAnnot(x$label), x$x, x$y, :
## 'mbcsToSbcs'里转换'濂藉弸鏁伴噺'出错:<b0>代替了dot
\end{verbatim}

\begin{verbatim}
## Warning in grid.Call(C_textBounds, as.graphicsAnnot(x$label), x$x, x$y, :
## 'mbcsToSbcs'里转换'濂藉弸鏁伴噺'出错:<e9>代替了dot
\end{verbatim}

\begin{verbatim}
## Warning in grid.Call(C_textBounds, as.graphicsAnnot(x$label), x$x, x$y, :
## 'mbcsToSbcs'里转换'濂藉弸鏁伴噺'出错:<87>代替了dot
\end{verbatim}

\begin{verbatim}
## Warning in grid.Call(C_textBounds, as.graphicsAnnot(x$label), x$x, x$y, :
## 'mbcsToSbcs'里转换'濂藉弸鏁伴噺'出错:<8f>代替了dot
\end{verbatim}

\begin{verbatim}
## Warning in grid.Call(C_textBounds, as.graphicsAnnot(x$label), x$x, x$y, :
## 'mbcsToSbcs'里转换'濂藉弸鏁伴噺'出错:<e5>代替了dot
\end{verbatim}

\begin{verbatim}
## Warning in grid.Call(C_textBounds, as.graphicsAnnot(x$label), x$x, x$y, :
## 'mbcsToSbcs'里转换'濂藉弸鏁伴噺'出错:<a5>代替了dot
\end{verbatim}

\begin{verbatim}
## Warning in grid.Call(C_textBounds, as.graphicsAnnot(x$label), x$x, x$y, :
## 'mbcsToSbcs'里转换'濂藉弸鏁伴噺'出错:<bd>代替了dot
\end{verbatim}

\begin{verbatim}
## Warning in grid.Call(C_textBounds, as.graphicsAnnot(x$label), x$x, x$y, :
## 'mbcsToSbcs'里转换'濂藉弸鏁伴噺'出错:<e5>代替了dot
\end{verbatim}

\begin{verbatim}
## Warning in grid.Call(C_textBounds, as.graphicsAnnot(x$label), x$x, x$y, :
## 'mbcsToSbcs'里转换'濂藉弸鏁伴噺'出错:<8f>代替了dot
\end{verbatim}

\begin{verbatim}
## Warning in grid.Call(C_textBounds, as.graphicsAnnot(x$label), x$x, x$y, :
## 'mbcsToSbcs'里转换'濂藉弸鏁伴噺'出错:<8b>代替了dot
\end{verbatim}

\begin{verbatim}
## Warning in grid.Call(C_textBounds, as.graphicsAnnot(x$label), x$x, x$y, :
## 'mbcsToSbcs'里转换'濂藉弸鏁伴噺'出错:<e6>代替了dot
\end{verbatim}

\begin{verbatim}
## Warning in grid.Call(C_textBounds, as.graphicsAnnot(x$label), x$x, x$y, :
## 'mbcsToSbcs'里转换'濂藉弸鏁伴噺'出错:<95>代替了dot
\end{verbatim}

\begin{verbatim}
## Warning in grid.Call(C_textBounds, as.graphicsAnnot(x$label), x$x, x$y, :
## 'mbcsToSbcs'里转换'濂藉弸鏁伴噺'出错:<b0>代替了dot
\end{verbatim}

\begin{verbatim}
## Warning in grid.Call(C_textBounds, as.graphicsAnnot(x$label), x$x, x$y, :
## 'mbcsToSbcs'里转换'濂藉弸鏁伴噺'出错:<e9>代替了dot
\end{verbatim}

\begin{verbatim}
## Warning in grid.Call(C_textBounds, as.graphicsAnnot(x$label), x$x, x$y, :
## 'mbcsToSbcs'里转换'濂藉弸鏁伴噺'出错:<87>代替了dot
\end{verbatim}

\begin{verbatim}
## Warning in grid.Call(C_textBounds, as.graphicsAnnot(x$label), x$x, x$y, :
## 'mbcsToSbcs'里转换'濂藉弸鏁伴噺'出错:<8f>代替了dot
\end{verbatim}

\begin{verbatim}
## Warning in grid.Call(C_textBounds, as.graphicsAnnot(x$label), x$x, x$y, :
## 'mbcsToSbcs'里转换'濂藉弸鏁伴噺'出错:<e5>代替了dot
\end{verbatim}

\begin{verbatim}
## Warning in grid.Call(C_textBounds, as.graphicsAnnot(x$label), x$x, x$y, :
## 'mbcsToSbcs'里转换'濂藉弸鏁伴噺'出错:<a5>代替了dot
\end{verbatim}

\begin{verbatim}
## Warning in grid.Call(C_textBounds, as.graphicsAnnot(x$label), x$x, x$y, :
## 'mbcsToSbcs'里转换'濂藉弸鏁伴噺'出错:<bd>代替了dot
\end{verbatim}

\begin{verbatim}
## Warning in grid.Call(C_textBounds, as.graphicsAnnot(x$label), x$x, x$y, :
## 'mbcsToSbcs'里转换'濂藉弸鏁伴噺'出错:<e5>代替了dot
\end{verbatim}

\begin{verbatim}
## Warning in grid.Call(C_textBounds, as.graphicsAnnot(x$label), x$x, x$y, :
## 'mbcsToSbcs'里转换'濂藉弸鏁伴噺'出错:<8f>代替了dot
\end{verbatim}

\begin{verbatim}
## Warning in grid.Call(C_textBounds, as.graphicsAnnot(x$label), x$x, x$y, :
## 'mbcsToSbcs'里转换'濂藉弸鏁伴噺'出错:<8b>代替了dot
\end{verbatim}

\begin{verbatim}
## Warning in grid.Call(C_textBounds, as.graphicsAnnot(x$label), x$x, x$y, :
## 'mbcsToSbcs'里转换'濂藉弸鏁伴噺'出错:<e6>代替了dot
\end{verbatim}

\begin{verbatim}
## Warning in grid.Call(C_textBounds, as.graphicsAnnot(x$label), x$x, x$y, :
## 'mbcsToSbcs'里转换'濂藉弸鏁伴噺'出错:<95>代替了dot
\end{verbatim}

\begin{verbatim}
## Warning in grid.Call(C_textBounds, as.graphicsAnnot(x$label), x$x, x$y, :
## 'mbcsToSbcs'里转换'濂藉弸鏁伴噺'出错:<b0>代替了dot
\end{verbatim}

\begin{verbatim}
## Warning in grid.Call(C_textBounds, as.graphicsAnnot(x$label), x$x, x$y, :
## 'mbcsToSbcs'里转换'濂藉弸鏁伴噺'出错:<e9>代替了dot
\end{verbatim}

\begin{verbatim}
## Warning in grid.Call(C_textBounds, as.graphicsAnnot(x$label), x$x, x$y, :
## 'mbcsToSbcs'里转换'濂藉弸鏁伴噺'出错:<87>代替了dot
\end{verbatim}

\begin{verbatim}
## Warning in grid.Call(C_textBounds, as.graphicsAnnot(x$label), x$x, x$y, :
## 'mbcsToSbcs'里转换'濂藉弸鏁伴噺'出错:<8f>代替了dot
\end{verbatim}

\begin{verbatim}
## Warning in grid.Call(C_textBounds, as.graphicsAnnot(x$label), x$x, x$y, :
## 'mbcsToSbcs'里转换'濂藉弸鏁伴噺'出错:<e5>代替了dot
\end{verbatim}

\begin{verbatim}
## Warning in grid.Call(C_textBounds, as.graphicsAnnot(x$label), x$x, x$y, :
## 'mbcsToSbcs'里转换'濂藉弸鏁伴噺'出错:<a5>代替了dot
\end{verbatim}

\begin{verbatim}
## Warning in grid.Call(C_textBounds, as.graphicsAnnot(x$label), x$x, x$y, :
## 'mbcsToSbcs'里转换'濂藉弸鏁伴噺'出错:<bd>代替了dot
\end{verbatim}

\begin{verbatim}
## Warning in grid.Call(C_textBounds, as.graphicsAnnot(x$label), x$x, x$y, :
## 'mbcsToSbcs'里转换'濂藉弸鏁伴噺'出错:<e5>代替了dot
\end{verbatim}

\begin{verbatim}
## Warning in grid.Call(C_textBounds, as.graphicsAnnot(x$label), x$x, x$y, :
## 'mbcsToSbcs'里转换'濂藉弸鏁伴噺'出错:<8f>代替了dot
\end{verbatim}

\begin{verbatim}
## Warning in grid.Call(C_textBounds, as.graphicsAnnot(x$label), x$x, x$y, :
## 'mbcsToSbcs'里转换'濂藉弸鏁伴噺'出错:<8b>代替了dot
\end{verbatim}

\begin{verbatim}
## Warning in grid.Call(C_textBounds, as.graphicsAnnot(x$label), x$x, x$y, :
## 'mbcsToSbcs'里转换'濂藉弸鏁伴噺'出错:<e6>代替了dot
\end{verbatim}

\begin{verbatim}
## Warning in grid.Call(C_textBounds, as.graphicsAnnot(x$label), x$x, x$y, :
## 'mbcsToSbcs'里转换'濂藉弸鏁伴噺'出错:<95>代替了dot
\end{verbatim}

\begin{verbatim}
## Warning in grid.Call(C_textBounds, as.graphicsAnnot(x$label), x$x, x$y, :
## 'mbcsToSbcs'里转换'濂藉弸鏁伴噺'出错:<b0>代替了dot
\end{verbatim}

\begin{verbatim}
## Warning in grid.Call(C_textBounds, as.graphicsAnnot(x$label), x$x, x$y, :
## 'mbcsToSbcs'里转换'濂藉弸鏁伴噺'出错:<e9>代替了dot
\end{verbatim}

\begin{verbatim}
## Warning in grid.Call(C_textBounds, as.graphicsAnnot(x$label), x$x, x$y, :
## 'mbcsToSbcs'里转换'濂藉弸鏁伴噺'出错:<87>代替了dot
\end{verbatim}

\begin{verbatim}
## Warning in grid.Call(C_textBounds, as.graphicsAnnot(x$label), x$x, x$y, :
## 'mbcsToSbcs'里转换'濂藉弸鏁伴噺'出错:<8f>代替了dot
\end{verbatim}

\begin{verbatim}
## Warning in grid.Call(C_textBounds, as.graphicsAnnot(x$label), x$x, x$y, :
## 'mbcsToSbcs'里转换'濂藉弸鏁伴噺'出错:<e5>代替了dot
\end{verbatim}

\begin{verbatim}
## Warning in grid.Call(C_textBounds, as.graphicsAnnot(x$label), x$x, x$y, :
## 'mbcsToSbcs'里转换'濂藉弸鏁伴噺'出错:<a5>代替了dot
\end{verbatim}

\begin{verbatim}
## Warning in grid.Call(C_textBounds, as.graphicsAnnot(x$label), x$x, x$y, :
## 'mbcsToSbcs'里转换'濂藉弸鏁伴噺'出错:<bd>代替了dot
\end{verbatim}

\begin{verbatim}
## Warning in grid.Call(C_textBounds, as.graphicsAnnot(x$label), x$x, x$y, :
## 'mbcsToSbcs'里转换'濂藉弸鏁伴噺'出错:<e5>代替了dot
\end{verbatim}

\begin{verbatim}
## Warning in grid.Call(C_textBounds, as.graphicsAnnot(x$label), x$x, x$y, :
## 'mbcsToSbcs'里转换'濂藉弸鏁伴噺'出错:<8f>代替了dot
\end{verbatim}

\begin{verbatim}
## Warning in grid.Call(C_textBounds, as.graphicsAnnot(x$label), x$x, x$y, :
## 'mbcsToSbcs'里转换'濂藉弸鏁伴噺'出错:<8b>代替了dot
\end{verbatim}

\begin{verbatim}
## Warning in grid.Call(C_textBounds, as.graphicsAnnot(x$label), x$x, x$y, :
## 'mbcsToSbcs'里转换'濂藉弸鏁伴噺'出错:<e6>代替了dot
\end{verbatim}

\begin{verbatim}
## Warning in grid.Call(C_textBounds, as.graphicsAnnot(x$label), x$x, x$y, :
## 'mbcsToSbcs'里转换'濂藉弸鏁伴噺'出错:<95>代替了dot
\end{verbatim}

\begin{verbatim}
## Warning in grid.Call(C_textBounds, as.graphicsAnnot(x$label), x$x, x$y, :
## 'mbcsToSbcs'里转换'濂藉弸鏁伴噺'出错:<b0>代替了dot
\end{verbatim}

\begin{verbatim}
## Warning in grid.Call(C_textBounds, as.graphicsAnnot(x$label), x$x, x$y, :
## 'mbcsToSbcs'里转换'濂藉弸鏁伴噺'出错:<e9>代替了dot
\end{verbatim}

\begin{verbatim}
## Warning in grid.Call(C_textBounds, as.graphicsAnnot(x$label), x$x, x$y, :
## 'mbcsToSbcs'里转换'濂藉弸鏁伴噺'出错:<87>代替了dot
\end{verbatim}

\begin{verbatim}
## Warning in grid.Call(C_textBounds, as.graphicsAnnot(x$label), x$x, x$y, :
## 'mbcsToSbcs'里转换'濂藉弸鏁伴噺'出错:<8f>代替了dot
\end{verbatim}

\begin{verbatim}
## Warning in grid.Call(C_textBounds, as.graphicsAnnot(x$label), x$x, x$y, :
## 'mbcsToSbcs'里转换'濂藉弸鏁伴噺'出错:<e5>代替了dot
\end{verbatim}

\begin{verbatim}
## Warning in grid.Call(C_textBounds, as.graphicsAnnot(x$label), x$x, x$y, :
## 'mbcsToSbcs'里转换'濂藉弸鏁伴噺'出错:<a5>代替了dot
\end{verbatim}

\begin{verbatim}
## Warning in grid.Call(C_textBounds, as.graphicsAnnot(x$label), x$x, x$y, :
## 'mbcsToSbcs'里转换'濂藉弸鏁伴噺'出错:<bd>代替了dot
\end{verbatim}

\begin{verbatim}
## Warning in grid.Call(C_textBounds, as.graphicsAnnot(x$label), x$x, x$y, :
## 'mbcsToSbcs'里转换'濂藉弸鏁伴噺'出错:<e5>代替了dot
\end{verbatim}

\begin{verbatim}
## Warning in grid.Call(C_textBounds, as.graphicsAnnot(x$label), x$x, x$y, :
## 'mbcsToSbcs'里转换'濂藉弸鏁伴噺'出错:<8f>代替了dot
\end{verbatim}

\begin{verbatim}
## Warning in grid.Call(C_textBounds, as.graphicsAnnot(x$label), x$x, x$y, :
## 'mbcsToSbcs'里转换'濂藉弸鏁伴噺'出错:<8b>代替了dot
\end{verbatim}

\begin{verbatim}
## Warning in grid.Call(C_textBounds, as.graphicsAnnot(x$label), x$x, x$y, :
## 'mbcsToSbcs'里转换'濂藉弸鏁伴噺'出错:<e6>代替了dot
\end{verbatim}

\begin{verbatim}
## Warning in grid.Call(C_textBounds, as.graphicsAnnot(x$label), x$x, x$y, :
## 'mbcsToSbcs'里转换'濂藉弸鏁伴噺'出错:<95>代替了dot
\end{verbatim}

\begin{verbatim}
## Warning in grid.Call(C_textBounds, as.graphicsAnnot(x$label), x$x, x$y, :
## 'mbcsToSbcs'里转换'濂藉弸鏁伴噺'出错:<b0>代替了dot
\end{verbatim}

\begin{verbatim}
## Warning in grid.Call(C_textBounds, as.graphicsAnnot(x$label), x$x, x$y, :
## 'mbcsToSbcs'里转换'濂藉弸鏁伴噺'出错:<e9>代替了dot
\end{verbatim}

\begin{verbatim}
## Warning in grid.Call(C_textBounds, as.graphicsAnnot(x$label), x$x, x$y, :
## 'mbcsToSbcs'里转换'濂藉弸鏁伴噺'出错:<87>代替了dot
\end{verbatim}

\begin{verbatim}
## Warning in grid.Call(C_textBounds, as.graphicsAnnot(x$label), x$x, x$y, :
## 'mbcsToSbcs'里转换'濂藉弸鏁伴噺'出错:<8f>代替了dot
\end{verbatim}

\begin{verbatim}
## Warning in grid.Call.graphics(C_text, as.graphicsAnnot(x$label), x$x, x
## $y, : 'mbcsToSbcs'里转换'濂藉弸鏁伴噺'出错:<e5>代替了dot
\end{verbatim}

\begin{verbatim}
## Warning in grid.Call.graphics(C_text, as.graphicsAnnot(x$label), x$x, x
## $y, : 'mbcsToSbcs'里转换'濂藉弸鏁伴噺'出错:<a5>代替了dot
\end{verbatim}

\begin{verbatim}
## Warning in grid.Call.graphics(C_text, as.graphicsAnnot(x$label), x$x, x
## $y, : 'mbcsToSbcs'里转换'濂藉弸鏁伴噺'出错:<bd>代替了dot
\end{verbatim}

\begin{verbatim}
## Warning in grid.Call.graphics(C_text, as.graphicsAnnot(x$label), x$x, x
## $y, : 'mbcsToSbcs'里转换'濂藉弸鏁伴噺'出错:<e5>代替了dot
\end{verbatim}

\begin{verbatim}
## Warning in grid.Call.graphics(C_text, as.graphicsAnnot(x$label), x$x, x
## $y, : 'mbcsToSbcs'里转换'濂藉弸鏁伴噺'出错:<8f>代替了dot
\end{verbatim}

\begin{verbatim}
## Warning in grid.Call.graphics(C_text, as.graphicsAnnot(x$label), x$x, x
## $y, : 'mbcsToSbcs'里转换'濂藉弸鏁伴噺'出错:<8b>代替了dot
\end{verbatim}

\begin{verbatim}
## Warning in grid.Call.graphics(C_text, as.graphicsAnnot(x$label), x$x, x
## $y, : 'mbcsToSbcs'里转换'濂藉弸鏁伴噺'出错:<e6>代替了dot
\end{verbatim}

\begin{verbatim}
## Warning in grid.Call.graphics(C_text, as.graphicsAnnot(x$label), x$x, x
## $y, : 'mbcsToSbcs'里转换'濂藉弸鏁伴噺'出错:<95>代替了dot
\end{verbatim}

\begin{verbatim}
## Warning in grid.Call.graphics(C_text, as.graphicsAnnot(x$label), x$x, x
## $y, : 'mbcsToSbcs'里转换'濂藉弸鏁伴噺'出错:<b0>代替了dot
\end{verbatim}

\begin{verbatim}
## Warning in grid.Call.graphics(C_text, as.graphicsAnnot(x$label), x$x, x
## $y, : 'mbcsToSbcs'里转换'濂藉弸鏁伴噺'出错:<e9>代替了dot
\end{verbatim}

\begin{verbatim}
## Warning in grid.Call.graphics(C_text, as.graphicsAnnot(x$label), x$x, x
## $y, : 'mbcsToSbcs'里转换'濂藉弸鏁伴噺'出错:<87>代替了dot
\end{verbatim}

\begin{verbatim}
## Warning in grid.Call.graphics(C_text, as.graphicsAnnot(x$label), x$x, x
## $y, : 'mbcsToSbcs'里转换'濂藉弸鏁伴噺'出错:<8f>代替了dot
\end{verbatim}

\includegraphics{lesson3_student-zh_-测试能否knit_files/figure-latex/unnamed-chunk-32-1.pdf}
***

\subsubsection{网页上的赞数}

注释:

\begin{Shaded}
\begin{Highlighting}[]
\KeywordTok{qplot}\NormalTok{(}\DataTypeTok{x =}\NormalTok{ www_likes, }\DataTypeTok{data =} \KeywordTok{subset}\NormalTok{(pf, }\OperatorTok{!}\KeywordTok{is.na}\NormalTok{(gender)),}
      \DataTypeTok{geom =} \StringTok{'freqpoly'}\NormalTok{, }\DataTypeTok{color =}\NormalTok{ gender) }\OperatorTok{+}
\StringTok{  }\KeywordTok{scale_x_continuous}\NormalTok{() }\OperatorTok{+}\StringTok{ }
\StringTok{  }\KeywordTok{scale_x_log10}\NormalTok{()}
\end{Highlighting}
\end{Shaded}

\begin{verbatim}
## Scale for 'x' is already present. Adding another scale for 'x', which
## will replace the existing scale.
\end{verbatim}

\begin{verbatim}
## Warning: Transformation introduced infinite values in continuous x-axis
\end{verbatim}

\begin{verbatim}
## `stat_bin()` using `bins = 30`. Pick better value with `binwidth`.
\end{verbatim}

\begin{verbatim}
## Warning: Removed 60935 rows containing non-finite values (stat_bin).
\end{verbatim}

\includegraphics{lesson3_student-zh_-测试能否knit_files/figure-latex/unnamed-chunk-33-1.pdf}

\begin{Shaded}
\begin{Highlighting}[]
\KeywordTok{by}\NormalTok{(pf}\OperatorTok{$}\NormalTok{www_likes, pf}\OperatorTok{$}\NormalTok{gender, sum)}
\end{Highlighting}
\end{Shaded}

\begin{verbatim}
## pf$gender: female
## [1] 3507665
## -------------------------------------------------------- 
## pf$gender: male
## [1] 1430175
\end{verbatim}

\begin{center}\rule{0.5\linewidth}{\linethickness}\end{center}

\subsubsection{箱线图}

注释:

\begin{Shaded}
\begin{Highlighting}[]
\KeywordTok{qplot}\NormalTok{(}\DataTypeTok{x =}\NormalTok{ gender, }\DataTypeTok{y =}\NormalTok{ friend_count,}
      \DataTypeTok{data =} \KeywordTok{subset}\NormalTok{(pf,}\OperatorTok{!}\KeywordTok{is.na}\NormalTok{(gender)), }
      \DataTypeTok{geom =} \StringTok{'boxplot'}\NormalTok{, }\DataTypeTok{color =}\NormalTok{ gender) }
\end{Highlighting}
\end{Shaded}

\includegraphics{lesson3_student-zh_-测试能否knit_files/figure-latex/unnamed-chunk-35-1.pdf}

\begin{Shaded}
\begin{Highlighting}[]
  \CommentTok{#scale_x_continuous(limits = c(0, 1000), breaks = seq(0, 1000, 50))}
\end{Highlighting}
\end{Shaded}

\hypertarget{-0--1000-}{%
\paragraph{调整代码,关注朋友数量在 0 到 1000
之间的用户。}\label{-0--1000-}}

\begin{Shaded}
\begin{Highlighting}[]
\KeywordTok{qplot}\NormalTok{(}\DataTypeTok{x =}\NormalTok{ gender, }\DataTypeTok{y =}\NormalTok{ friend_count,}
      \DataTypeTok{data =} \KeywordTok{subset}\NormalTok{(pf,}\OperatorTok{!}\KeywordTok{is.na}\NormalTok{(gender)), }
      \DataTypeTok{geom =} \StringTok{'boxplot'}\NormalTok{, }\DataTypeTok{color =}\NormalTok{ gender) }\OperatorTok{+}
\StringTok{  }\KeywordTok{scale_y_continuous}\NormalTok{(}\DataTypeTok{limits =} \KeywordTok{c}\NormalTok{(}\DecValTok{0}\NormalTok{, }\DecValTok{1000}\NormalTok{))}
\end{Highlighting}
\end{Shaded}

\begin{verbatim}
## Warning: Removed 2949 rows containing non-finite values (stat_boxplot).
\end{verbatim}

\includegraphics{lesson3_student-zh_-测试能否knit_files/figure-latex/unnamed-chunk-36-1.pdf}

\begin{Shaded}
\begin{Highlighting}[]
\KeywordTok{qplot}\NormalTok{(}\DataTypeTok{x =}\NormalTok{ gender, }\DataTypeTok{y =}\NormalTok{ friend_count,}
      \DataTypeTok{data =} \KeywordTok{subset}\NormalTok{(pf,}\OperatorTok{!}\KeywordTok{is.na}\NormalTok{(gender)), }
      \DataTypeTok{geom =} \StringTok{'boxplot'}\NormalTok{, }\DataTypeTok{color =}\NormalTok{ gender, }\DataTypeTok{ylim =} \KeywordTok{c}\NormalTok{(}\DecValTok{0}\NormalTok{, }\DecValTok{1000}\NormalTok{))}
\end{Highlighting}
\end{Shaded}

\begin{verbatim}
## Warning: Removed 2949 rows containing non-finite values (stat_boxplot).
\end{verbatim}

\includegraphics{lesson3_student-zh_-测试能否knit_files/figure-latex/unnamed-chunk-37-1.pdf}
以上两种方法实际上从计算中删除了数据点,所以更好的方法是使用coord
Cartesian层来设置y轴,

\begin{Shaded}
\begin{Highlighting}[]
\KeywordTok{qplot}\NormalTok{(}\DataTypeTok{x =}\NormalTok{ gender, }\DataTypeTok{y =}\NormalTok{ friend_count,}
      \DataTypeTok{data =} \KeywordTok{subset}\NormalTok{(pf,}\OperatorTok{!}\KeywordTok{is.na}\NormalTok{(gender)), }
      \DataTypeTok{geom =} \StringTok{'boxplot'}\NormalTok{, }\DataTypeTok{color =}\NormalTok{ gender) }\OperatorTok{+}
\StringTok{  }\KeywordTok{coord_cartesian}\NormalTok{(}\DataTypeTok{ylim =} \KeywordTok{c}\NormalTok{(}\DecValTok{0}\NormalTok{, }\DecValTok{1000}\NormalTok{))}
\end{Highlighting}
\end{Shaded}

\includegraphics{lesson3_student-zh_-测试能否knit_files/figure-latex/unnamed-chunk-38-1.pdf}

\begin{center}\rule{0.5\linewidth}{\linethickness}\end{center}

\subsubsection{箱线图、分位数和友情}

注释:

\begin{Shaded}
\begin{Highlighting}[]
\KeywordTok{by}\NormalTok{(pf}\OperatorTok{$}\NormalTok{friend_count, pf}\OperatorTok{$}\NormalTok{gender, summary)}
\end{Highlighting}
\end{Shaded}

\begin{verbatim}
## pf$gender: female
##    Min. 1st Qu.  Median    Mean 3rd Qu.    Max. 
##       0      37      96     242     244    4923 
## -------------------------------------------------------- 
## pf$gender: male
##    Min. 1st Qu.  Median    Mean 3rd Qu.    Max. 
##       0      27      74     165     182    4917
\end{verbatim}

\paragraph{一般来说,我们样本中哪些人更容易交朋友,男性还是女性?}

回复:女性 \#\#\#\# 写出可以验证答案的一些方法。 回复:

\begin{Shaded}
\begin{Highlighting}[]
\KeywordTok{qplot}\NormalTok{(}\DataTypeTok{x =}\NormalTok{ gender, }\DataTypeTok{y =}\NormalTok{ friendships_initiated,}
      \DataTypeTok{data =} \KeywordTok{subset}\NormalTok{(pf,}\OperatorTok{!}\KeywordTok{is.na}\NormalTok{(gender)), }
      \DataTypeTok{geom =} \StringTok{'boxplot'}\NormalTok{, }\DataTypeTok{color =}\NormalTok{ gender) }\OperatorTok{+}
\StringTok{  }\KeywordTok{coord_cartesian}\NormalTok{(}\DataTypeTok{ylim =} \KeywordTok{c}\NormalTok{(}\DecValTok{0}\NormalTok{,}\DecValTok{200}\NormalTok{))}
\end{Highlighting}
\end{Shaded}

\includegraphics{lesson3_student-zh_-测试能否knit_files/figure-latex/unnamed-chunk-40-1.pdf}

\begin{Shaded}
\begin{Highlighting}[]
\KeywordTok{by}\NormalTok{(pf}\OperatorTok{$}\NormalTok{friendships_initiated, pf}\OperatorTok{$}\NormalTok{gender, summary)}
\end{Highlighting}
\end{Shaded}

\begin{verbatim}
## pf$gender: female
##    Min. 1st Qu.  Median    Mean 3rd Qu.    Max. 
##     0.0    19.0    49.0   113.9   124.8  3654.0 
## -------------------------------------------------------- 
## pf$gender: male
##    Min. 1st Qu.  Median    Mean 3rd Qu.    Max. 
##     0.0    15.0    44.0   103.1   111.0  4144.0
\end{verbatim}

回复:

\begin{center}\rule{0.5\linewidth}{\linethickness}\end{center}

\subsubsection{得到逻辑}

注释:

\begin{Shaded}
\begin{Highlighting}[]
\KeywordTok{summary}\NormalTok{(pf}\OperatorTok{$}\NormalTok{mobile_likes)}
\end{Highlighting}
\end{Shaded}

\begin{verbatim}
##    Min. 1st Qu.  Median    Mean 3rd Qu.    Max. 
##     0.0     0.0     4.0   106.1    46.0 25111.0
\end{verbatim}

\begin{Shaded}
\begin{Highlighting}[]
\KeywordTok{summary}\NormalTok{(pf}\OperatorTok{$}\NormalTok{mobile_likes }\OperatorTok{>}\StringTok{ }\DecValTok{0}\NormalTok{)}
\end{Highlighting}
\end{Shaded}

\begin{verbatim}
##    Mode   FALSE    TRUE 
## logical   35056   63947
\end{verbatim}

\begin{Shaded}
\begin{Highlighting}[]
\NormalTok{pf}\OperatorTok{$}\NormalTok{mobile_check_in <-}\StringTok{ }\OtherTok{NA}
\NormalTok{pf}\OperatorTok{$}\NormalTok{mobile_check_in <-}\StringTok{ }\KeywordTok{ifelse}\NormalTok{(pf}\OperatorTok{$}\NormalTok{mobile_likes }\OperatorTok{>}\StringTok{ }\DecValTok{0}\NormalTok{, }\DecValTok{1}\NormalTok{, }\DecValTok{0}\NormalTok{)}
\KeywordTok{summary}\NormalTok{(pf}\OperatorTok{$}\NormalTok{mobile_check_in)}
\end{Highlighting}
\end{Shaded}

\begin{verbatim}
##    Min. 1st Qu.  Median    Mean 3rd Qu.    Max. 
##  0.0000  0.0000  1.0000  0.6459  1.0000  1.0000
\end{verbatim}

\begin{Shaded}
\begin{Highlighting}[]
\NormalTok{pf}\OperatorTok{$}\NormalTok{mobile_check_in <-}\StringTok{ }\KeywordTok{factor}\NormalTok{(pf}\OperatorTok{$}\NormalTok{mobile_check_in)}
\KeywordTok{summary}\NormalTok{(pf}\OperatorTok{$}\NormalTok{mobile_check_in)}
\end{Highlighting}
\end{Shaded}

\begin{verbatim}
##     0     1 
## 35056 63947
\end{verbatim}

\begin{Shaded}
\begin{Highlighting}[]
\KeywordTok{sum}\NormalTok{(pf}\OperatorTok{$}\NormalTok{mobile_check_in }\OperatorTok{==}\StringTok{ }\DecValTok{1}\NormalTok{) }\OperatorTok{/}\StringTok{ }\KeywordTok{length}\NormalTok{(pf}\OperatorTok{$}\NormalTok{mobile_check_in)}
\end{Highlighting}
\end{Shaded}

\begin{verbatim}
## [1] 0.6459097
\end{verbatim}

回复:

\begin{center}\rule{0.5\linewidth}{\linethickness}\end{center}

\subsubsection{分析一个变量}

思考:

点击 \textbf{KnitHTML} 查看你的成果和这节课的 html 页面、答案和注释!


\end{document}
